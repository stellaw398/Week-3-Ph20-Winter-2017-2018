\documentclass{article}
\usepackage{pdfpages}
\title{Ph20.2 - Introduction to Numerical Techniques: Numerical Integration}
\author{Stella Wang}
\date{24 January 2018}
\begin{document}
	\maketitle
\section{Simpson's formula}
\[I \simeq I_{simp} \equiv H\bigg ( \frac{f(a)}{6} + \frac{4f(c)}{6} + \frac{f(b)}{6} \bigg) ; \]
\[ \int_a^b f(x)dx = H \Bigg( \frac{f(x_0)}{6} + \frac{4f(x_1)}{6}  + \frac{f(x_2)}{6} \Bigg) + H \Bigg( \frac{f(x_2)}{6} + \frac{4f(x_3)}{6}  + \frac{f(x_4)}{6} \Bigg)+ \cdots + H \Bigg( \frac{f(x_{n-2})}{6} + \frac{4f(x_{n-1})}{6}  + \frac{f(x_n)}{6} \Bigg) \]
\[ \equiv \frac{h}{3} \Bigg( f(x_0) +4f(x_1) + 2f(x_2) + 4f(x_3) + 2f(x_4) + \cdots + 4f(x_{n-1}) + f(x_n) \Bigg) \]
\[ f(x) = f(a) + f'(a)(x-a) + \frac {f''(a)}{2!}(x-a)^2 + \frac {f'''(a)}{3!}(x-a)^3+ \frac{f^4(\eta)}{4!} \]
\[I = f(a)H + f'(a)\frac{H^2}{2!} + f''(a)\frac{H^3}{3!} + f'''(a)\frac{H^4}{4!} + f^4(\eta)\frac{H^5}{5!} \]
\[I_{simp} = H\bigg ( \frac{f(a)}{6} + \frac{4f(c)}{6} + \frac{f(b)}{6} \bigg) =\]
This is a plot of the convergence rate of the error for the trapezoid rule and Simpson's method for N ranging from 500 to 5000. The red curve represents the Simpson's method and the blue curve shows the trapezoid rule. The x axis is in log scale in order to show that the trapezoid rule does not reach the simpson's method error rapidly. The second graph shows just the Simpson's method for N 500 to 3000. From this graph it is more clear that the Simpson's method error stops decreasing around 2000. This is most likely because of the limits of the machine when the subintervals become that small. 
\includepdf[pages={1}]{globalerrorcomparison.pdf}
\includepdf[pages={1}]{globalerrorcomparison2.pdf}
\section{Relative difference}
Using the relative  difference formula given, I found that for $f(x)= e^x$ the relative difference after 8 iterations is less than $10^{-9}$. The relative difference found for $f(x)=log(x+1)$ was also less than $10^{-9}$ after 8 iterations.

\section{Comparison with scipy.integrate}
The scipy.integrate.quad and scipy.integrate.romberg functions for $f(x)=e^x$ return $(1.7182818284590453, 1.9076760487502457*10^-{14})$ and $1.782818284590782$. These values are comparable to the numbers we obtain from the trapezoid rule and Simpson's Method for N = 1000: $1.7182819716491953$ and $1.7182818284590549$. We can see that at least for Simpson's Method, it is accurate up to the $13th$ decimal place, and is within the error bound of scipy.integrate.quad. 

\end{document}